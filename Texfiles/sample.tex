
\subsubsection{Data}
\label{subsec:data}
The data recorded in 2015-2018 with integrated luminosity of 140 \ifb (full Run 2 data) is used in this study. The data samples are processed through the un-skimmed DAOD\_JETM1 derivation scheme in order to obtain multi-jet events. The lowest un-prescaled small-$R$ single-jet trigger is employed for this analysis. The jet \pt~threshold for the trigger in this analysis is 420 GeV, keeping the selection consistent across years, together with additional requirements that ensure events of good qualities are used. The additional selections are:
\begin{itemize}
	\item Good Run List (GRL): Make sure a steady state of all relevant detectors so that physics processes recorded by them are good.
	\item LAr: Liquid Argon Calorimeter error rejected.
	\item Tile: Tile Calorimeter error rejected.
	\item SCT: SCT single event upsets rejected.
	\item Core: Incomplete event build rejected.
	\item Primary Vertex: the highest $\sum\pt^{2}(trk)$ vertex has at least two tracks associated with it
	\item Trigger: Passes the lowest unprescaled single-jet trigger, HLT\_j420
\end{itemize}

Additional kinematic selection criteria are  discussed in Section~\ref{sec:Obj-event}.


\subsubsection{Monte Carlo simulation}
\label{subsec:MC}

For this calibration, multi-jet events are generated and modelled with several MC simulations, processed through the same DAOD\_JETM1 derivation scheme. For the nominal result, \pythia8.230 MC generator is used with leading-order (LO) matrix element (ME) for dijet production. Parton density functions (PDFs) are considered for systematic uncertainties evaluation as the \nnpdftwoLO PDF set is used for \pythia8.230.  Alternative samples with different choices of parton shower modelling, ME generation,  and the simulation of the multi-parton interactions are included to estimate the systematic uncertainties.

Two set of MC samples generated using \sherpa2.2.5 are used with the same ME for the (2$\rightarrow$2) process at LO, to provide the uncertainties of hadronization modeling. The \ctten PDF sets are included in both \sherpa samples where one based on the cluster hadronization whereas the other used \sherpa interface to the Lund string fragmentation model as \pythia8.230.

Two set of MC samples generated using \herwig7.1.3 are used for parton shower uncertainties as one uses angular ordering shower whereas the other one uses dipole shower. These samples are produced at next-to-leading order (NLO) with a PDF set of \mmht.


Another set of multijet samples that produced with \powheg interfaced to \pythia at NLO accuracy is employed with \nnpdftwo LO PDF set, to estimate the effects from the ME uncertainty as different perturbative scales in the ME and parton distribution functions are included. The renormalization and factorisation scales are set to the \pt~of the underlying Born configuration. These samples included different perturbative scales in the ME and parton distribution functions are used for the estimation of ME uncertainty.


%In this study, several multi-jet samples after JETM1 DAOD un-skimmed derivation with different modelings are used. %The parton shower modeling of the Monte Carlo~(MC) simulation is \pythia8 \cite{Sjostrand:2014zea} and \sherpa \cite{Gleisberg:2008ta}. %There are Pythia8 and Sherpa 2.2.5 samples which include both Scale and Parton Density Function~(PDF) variations. To flatten statistics across lower-pT slices, where the drop in cross section with increasing pT is greatest, weighted (JZW) filtering has been applied to the four lowest-pT slices. Slices JZ5 - JZ9plus are with JZ slicing. 

%There are 2 sets of Sherpa samples with the same Matrix Element~(ME) and shower configurations but different hadronization, which can be used for the estimation of uncertainties coming from fragmentation.

%Apart from these, there are Herwig 7.1 NLO~\cite{Bellm:2015jjp} samples (from JZ1 to JZ9plus) including scale variations from hard scattering and shower. There are 2 set of Herwig samples with same ME and hadronization but different types of showers, which can be used to investigate the effects of the parton shower modeling.
A list of the MC samples used is given in table~\ref{tab:MC}.




\begin{table}[H]
%\tiny
%\renewcommand{\arraystretch}{1.2}
\begin{center}
\begin{tabular}{ |c |c |c| c | c |}
  \hline
  PDF set & Generator      & Cross-section& Parton shower & Hadronisation \\ 
  \hline

\nnpdftwo              & \pythia8.230
         & LO            & \pt-ordered  & String \\
  \hline                                                                                                                                                              
\ctten                     & \sherpa2.2.5   & LO           & \pt-ordered  &Cluste      \\
  \hline
\ctten                     & \sherpa2.2.5   & LO           & \pt-ordered  & String \\
  \hline
 \mmht
  & \herwig7.1.3   & NLO           & Dipole & Cluster   \\
  \hline
\mmht
  & \herwig7.1.3   & NLO           & Angular-ordered & Cluster   \\
 \hline
\nnpdftwo
& Powheg+\pythia  & NLO         &   \pt-ordered  & String    \\
  \hline
\end{tabular}
\caption{The MC simulation used for the multi-jet processes in this calibration. %
	The PDF sets, generators for a hard process, the order in $\alpha_{\mrm s}$ of cross-section calculations and the simulator of parton showers, %
	and hadronisation are shown. }
\footnotesize
\label{tab:MC}
\end{center}
\end{table}




