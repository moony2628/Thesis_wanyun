
%\begin{figure}[htb]
%  \centering
%  \subfloat[]{\label{fig:QG-ROCa}\includegraphics[width=0.45\textwidth]{Figures/JetHL_ROC/jetWtrk_TightPt500_Pt0100-0150_ROC}} \quad
%  \subfloat[]{\label{fig:QG-ROCb}\includegraphics[width=0.45\textwidth]{Figures/JetHL_ROC/jetWtrk_TightPt500_Pt0800-1000_ROC}}
%  \caption[]{
%    \label{fig:QG-ROC}
%    }
%\end{figure}

\section{Quark/gluon separation variable}
\label{sec:QG-var}

In order to distinguish quarks from gluons, %
the information of the track activity inside a jet is important %
because the color factor of gluons is larger than that of quarks by factor 9/4~("Casimir ratio"), %
which makes gluons emit more particles in the hadronization than quarks. %
Thus, a gluon has more charged tracks in its jet and the jet width is larger than that of a quark-jet.

Here, jet width computed from the associated tracks \wtrk is used as a quark/gluon separation~("q/g separation") variable, %
which is a track-\pt-weighted width of the jet divided by the scalar sum of track transverse momenta. %
It is defined as %
\begin{equation}
  \wtrk = \frac{ \sum_{\mrm{trk}\in\mrm{jet}}\pttrk\Delta R_{\mrm{track,jet}} }{ \sum_{\mrm{trk}\in\mrm{jet}}\pttrk },
\end{equation}
where \pttrk is a \pt of a charged track reconstructed by the inner detector~(ID) and %
$\Delta R_{\mrm{track,jet}}$ is a distance in the \eta-\phi plane between the track and the jet axis. %
This variable is insensitive to the track inefficiency because it is defined as a ratio. %only and it has only small \eta dependency. %
The charged tracks used here are required to have $\pttrk>1\GeV$ and identified by the "TightPrimary" criteria~\cite{TrackPerformance} %
in order to remove fake tracks which make \etaX-dependence in \wtrk because high \eta range is under severer pileup environment %
and more fake tracks can be reproduced there. %

The calibration for this variable and estimation of its uncertainties are necessary %
since such a jet substructure information is not used in the conventional SUSY searches and %
the mis-modeling of the simulation, especially in a gluon, is known in the previous study for q/g separation in Run1~\cite{QGRun1}. %
The calibration of the q/g separation variable is performed by applying binned scale factor in the simulation for each quark and gluon jets, respectively.
The scale factor is obtained from the \wtrk distributions in quark and gluon jets from data in order to match the shape of the simulation to that of the data, %
in which the shape is obtained in each jet \pt range because it depends on jet \pt strongly. %
The jet used in this calibration is restricted to jets with $\pt>40\GeV$ and $\abseta<2.1$. %
The slightly tighter \abseta requirement than an usual jet identification~($\abseta<2.8$) is %
in order to avoid \abseta dependence of \wtrk caused by the ID coverage~(\abseta<2.5). %

\section{Method to extract quark/gluon from data}
\label{sec:QG-method}

To extract the shape of \wtrk distributions for quark jets and gluon jets from data separately, %
a "matrix method" of two samples with different quark/gluon fractions is used. %
The matrix method can extract pure quark or gluon jets from quark-enriched and gluon-enriched samples %
under the assumption that the two sample have the same shape of \wtrk distributions in each of quark and gluon jets. %
The matrix method is performed for each \pt bin defined in \Tab{tab:QG-ptbinning} and %
the quark-enriched and gluon-enriched samples are different between lower \pt ranges and higher \pt ranges. %

\begin{table}[hptb]
\centering
\caption{
  The \pt range division for the calibration of \wtrk and samples used in extraction of pure quark and gluon jets. %
}
\begin{tabular}{|c|c|c|c|c|c|c|c|c|c|c|c|c|c|}
 \hline
 \multicolumn{14}{|c|}{Lower \pt bin boundary [\GeVX] } \\ \hline
  40 & 60 & 100 & 150 & 200 & 300 & 400        & 500 & 600 & 800 & 1000 & 1200 & 1500 & 2000 \\ \hline
 \multicolumn{7}{|c}{Z+jets \& multi-jet samples} & \multicolumn{7}{|c|}{\multirow{2}{*}{Higher \& lower |\etaX| jet samples in multi-jet}} \\ 
 \multicolumn{7}{|c}{(2-process extraction)} & \multicolumn{7}{|c|}{} \\ \hline
\end{tabular}
\label{tab:QG-ptbinning}
\end{table}

%For the lower \pt ranges~(40--500\GeV), the quark-enriched sample is obtained from the leading \pt jet in Z+jets events %
%and the gluon-enriched sample is obtained from the leading \pt jet in multi-jet events. %
%The quark in Z+jets is an associated quark in the Feynman diagram given in \Figss{fig:bkg-V}. %
%The gluon in multi-jet comes from a gluon emission in QCD process. %
%This extraction method is referred to as "2-process extraction" hereafter. %

\begin{figure}[htb]
  \centering
  \subfloat[]{\label{fig:bkg-Vb}\includegraphics[width=0.3\textwidth]{Figures/Feynman/V+q_tchan}}
  \quad
  \subfloat[]{\label{fig:bkg-Va}\includegraphics[width=0.3\textwidth]{Figures/Feynman/V+q}}
  \caption[]{
    Main process of one vector boson~($V=Z$) plus one parton~(quark~$q$ or gluon~$g$) at tree-level. %
    }
  \label{fig:bkg-V}
\end{figure}



For the higher \pt ranges~(500--2000\GeV), the quark-enriched sample is obtained from higher \abseta jet of the leading two jets in multi-jet events %
and the gluon-enriched sample is obtained from lower \abseta jet of the leading two jets in the multi-jet events. %
The higher \abseta jet comes from high $x$~(momentum fraction) of a PDF. %
Since the PDF in the high-$x$ range has high probability of valence-quarks, %
the higher \abseta jet sample has more quarks. %
In contrast, the lower \abseta jet has lower $x$ and more gluons. %

The matrix in 2-process extraction is given as 
\begin{eqnarray}
\label{eq:QG-matrix}
\left(\begin{array}{c} 
    p_{\mrm{Z+jets}}(\wtrk)\\ 
    p_{\mrm{Multi-jet}}(\wtrk)\\ 
    \end{array}\right)
    &=&
    \underbrace{
\left(\begin{array}{cc}
    f_{\mrm{Z+jets,Q}}&f_{\mrm{Z+jets,G}}\\
    f_{\mrm{Multi-jet,Q}}&f_{\mrm{Multi-jet,G}}\\
    \end{array}\right)
  }_{\scalebox{1}{$\equiv F$}}
\left(\begin{array}{c} 
    p_\mrm{Q}(\wtrk)\\ 
    p_\mrm{G}(\wtrk)\\ 
    \end{array}\right)
    \\
\label{eq:QG-invmatrix}
\Leftrightarrow
\left(\begin{array}{c} 
    p_\mrm{Q}(\wtrk)\\ 
    p_\mrm{G}(\wtrk)\\ 
    \end{array}\right)
    &=& F^{-1}
\left(\begin{array}{c} 
    p_{\mrm{Z+jets}}(\wtrk)\\ 
    p_{\mrm{Multi-jet}}(\wtrk)\\ 
    \end{array}\right),
\end{eqnarray}
and the matrix using the higher and lower \abseta jets is 
\begin{eqnarray}
\label{eq:QG-matrix2}
\left(\begin{array}{c} 
    p_{\mrm{H}}(\wtrk)\\ 
    p_{\mrm{L}}(\wtrk)\\ 
    \end{array}\right)
    &=&
    \underbrace{
\left(\begin{array}{cc}
    f_{\mrm{H,Q}}&f_{\mrm{H,G}}\\
    f_{\mrm{L,Q}}&f_{\mrm{L,G}}\\
    \end{array}\right)
  }_{\scalebox{1}{$\equiv F$}}
\left(\begin{array}{c} 
    p_\mrm{Q}(\wtrk)\\ 
    p_\mrm{G}(\wtrk)\\ 
    \end{array}\right)
    \\
\label{eq:QG-invmatrix2}
\Leftrightarrow
\left(\begin{array}{c} 
    p_\mrm{Q}(\wtrk)\\ 
    p_\mrm{G}(\wtrk)\\ 
    \end{array}\right)
    &=& F^{-1}
\left(\begin{array}{c} 
    p_{\mrm{H}}(\wtrk)\\ 
    p_{\mrm{L}}(\wtrk)\\ 
    \end{array}\right),
\end{eqnarray}
where $p_{\mrm{Q,G}}(\wtrk)$ show \wtrk distributions in pure quark and gluon-jet samples, 
$p_{\mrm{Z+jets/Multi-jet/H/L}}(\wtrk)$ show \wtrk distributions in Z+jets, multi-jet, higher \abseta jet, and lower \abseta jet samples, respectively, 
and $f_{\mrm{X,Q/G}}$ are fractions of quark and gluon in a sample X. %
Between \Eqns{eq:QG-matrix}{eq:QG-invmatrix} or \Eqns{eq:QG-matrix2}{eq:QG-invmatrix2}, %
the inverse matrix of $F$ is calculated and used to extract pure quark/gluon~($p_{\mrm{Q,G}}$). %
The distribution of quark/gluon-enriched sample is obtained from data and the fraction of quarks and gluons in them are obtained from simulations. %
This matrix is computed in each \wtrk bin and each jet \pt range.

To avoid a systematic uncertainty coming from the parton shower modeling, %
this calibration is performed for \pythia8 and \sherpa~separately. %
%For \pythia8, both of the $Z$+jets and multi-jet MCs are \pythia8. %
For \sherpa, the $Z$+jets MC is \sherpa2.2.1 and the multi-jet MC is \sherpa2.1.1. %
This difference will be taken into account in the systematic uncertainties. %
In \Sectrange{sec:QG-method}{sec:QG-SF}, the figures using \pythia8~(\sherpa) show results %
with  NNPDF30~(NNPDF30) and NNPDF23LO~(CT10) PDF sets for 2-process extraction and for higher/lower \abseta jet extraction, respectively. %

%In \Sectrange{sec:QG-var}{sec:QG-SF}, the figures show the result with the \pythia8 parton shower modeling
%and the NNPDF30 PDF set for 2-process extraction or NNPDF23LO set for higher/lower \abseta jet. %
%A14 tuned NNPDF23LO set. %

The selections for all the samples used here are summarized in \Tab{tab:QG-sample}. %
The pure Z+jets events are obtained by requirements of two opposite charge leptons %
and Z mass cut between 75\GeV and 105\GeV in the invariant mass of the two leptons. %
The multi-jet sample consists of no-lepton events. %
%There is $\abseta<2.1$ requirement because the further higher \abseta jet has \eta dependence in \wtrk %
%and the ID is covered up to $\abseta=2.5$. %
There is $\abseta<2.1$ requirement because the ID covers \abseta up to $2.5$. %


\begin{table}[hptb]
\centering
\caption{
  The selections to retrieve quark/gluon-enriched samples.
  The requirements in the upper half are in order to obtain Z+jets/multi-jet events.
  The ones in the lower half are to enrich quark jets or gluon jets.
  "$j_i$" represents the $i$-th jet in \ptX-ordering.
}
\begin{tabular}{l|l|c|c|cc}
 \hline
 \multicolumn{2}{l|}{\multirow{3}{*}{Selection}}& \multirow{3}{*}{Z+jets sample} & \multicolumn{3}{c}{Multi-jet sample}  \\ 
 \hhline{~~~---}
 \multicolumn{2}{l|}{} &                        & For 2-process & Higher |\etaX| & Lower |\etaX| \\ 
 \multicolumn{2}{l|}{} &                        & extraction    & jet sample  & jet sample \\ \hline
\parbox[t]{2mm}{\multirow{7}{*}{\rotatebox[origin=c]{90}{Preselection}}}
 & Trigger   & Single lepton trigger & \multicolumn{3}{c}{Or of single jet triggers} \\ 
 & Object    & Two opposite charge leptons & \multicolumn{3}{c}{No lepton}        \\ 
 & $\mll$        & $75<\mll<105 \GeV$ & \multicolumn{3}{c}{-} \\
 & Number of jets        &  $\geq1$ & \multicolumn{3}{c}{$\geq2$} \\ \hhline{~~~---}
 & $b$-veto              & $j_1\neq b$-jet   & $j_1\neq b$-jet & \multicolumn{2}{c}{$j_1\neq b$-jet and $j_2\neq b$-jet} \\ 
 & $|\eta(j_1)|$         &  $<2.1$  & $<2.1$ & \multicolumn{2}{c}{$<2.1$}   \\ 
 & $|\eta(j_2)|$         &  -       & -      & \multicolumn{2}{c}{$<2.1$}   \\ \hline
\parbox[t]{2mm}{\multirow{8}{*}{\rotatebox[origin=c]{90}{Quark/gluon enhance}}}
 & Target parton         &  Quark   & Gluon  & Quark  & Gluon       \\ \hhline{~-----}
 & $\pt(Z)/\pt(j_2)$     &  <1.5    & -    & \multicolumn{2}{c}{-}       \\ 
 & $\pt(j_1)/\pt(j_2)$   &  -       & -    & \multicolumn{2}{c}{$<1.5$} \\ 
 & $\pt(j_2)$            &  $<$max$\kakko{30\GeV, 0.5\pt(Z)}$ & $>20\GeV$ & \multicolumn{2}{c}{$>20\GeV$}  \\
 & $\Delta\phi(Z,j_1)$   &  $>2.5$  & -    & \multicolumn{2}{c}{-}  \\ 
 & $\Delta\phi(j_1,j_2)$ &  -       & $>2.5$         & \multicolumn{2}{c}{-}  \\ 
 & $|\eta(j_1)|$         &  -       & $<|\eta(j_2)|$ & \multicolumn{2}{c}{-}  \\ 
 & Used jet in $j_1$ or $j_2$ & Only $j_1$ & Only $j_1$ & Higher $|\etaX|$ jet & Lower $|\etaX|$ jet \\
 \hline
\end{tabular}
\label{tab:QG-sample}
\end{table}

The jet \pt distribution for each sample with \pythia8 is shown in \Figs{fig:QG-2samplePt}{fig:QG-jetHLPt}. %
The partonic flavor label~(quark[$u$, $d$, $s$, or $c$], gluon, $b$-quark or "other") of a jet in the simulation is defined %
by a flavor of the highest-energy parton in the parton shower %
within $\Delta R=0.4$, which is equal to the radius parameter of the jet algorithm, with the jet.
The fractions of quarks and gluons in each sample are shown in \Figss{fig:QG-Fmc}.
In a low \pt range~($<500\GeV$), the quark fraction of Z+jets is high~($\sim75\%$) %
and the difference between the quark fractions in Z+jets and multi-jet is large~($30$--$50\%$), %
but the quark fraction of higher \abseta jet is low~($\lesssim 50\%$). % 
Thus, the Z+jets and multi-jet are used as quark/gluon-enriched samples in the low \pt range. %
In a high \pt range~($>500\GeVX$), higher \abseta jet has a large fraction of quarks~($>60\%$). %
Thus, in the low \pt range, the higher/lower \abseta jet samples are used\footnote{
The low statistics of Z+jets sample in the higher \pt range is another reason to use the higher/lower \abseta jet samples. %
}. %
%The fractions are obtained from \pythia8 and \sherpa, separately. %
The difference between the MC event generators in the fractions is known to be small in the previous study~\cite{ref21}. %
In the whole \pt range, $b$-quark jets and jets labeled "other" exist, %
but it is suppressed to be lower than a few \%, which can be ignored. %
The jets labeled "other" are jets mainly originating from pileup. % or a jet that unable to be labeled. %

\begin{figure}[htb]
  \centering
  \subfloat[Z+jets~(\pythia8)   ]{\label{fig:QG-2samplePta}\includegraphics[width=0.45\textwidth]{Figures/ZCut0_jetPt0B}} \quad
  \subfloat[Multi+jet~(\pythia8)]{\label{fig:QG-2samplePtb}\includegraphics[width=0.45\textwidth]{Figures/dijetCut0_jetPt0B}}
  \caption[]{
    The \pt distribution of the leading jets with \pythia8 MC in \subref{fig:QG-2samplePta} Z+jets 
    and \subref{fig:QG-2samplePtb} multi-jet for 2-sample process extraction. %
    Data for \subref{fig:QG-2samplePta} is 32.9\ifb in 2016 and that for \subref{fig:QG-2samplePtb} is 3.2\ifb in 2015.
    The normalization of the simulation is decided by cross-section.
    \label{fig:QG-2samplePt}
    }
\end{figure}

\begin{figure}[htb]
  \centering
  \subfloat[Higher \abseta jet~(\pythia8)]{\label{fig:QG-jetHLPta}\includegraphics[width=0.45\textwidth]{Figures/jetHLNoSkimCut_jetPtH}} \quad
  \subfloat[Lower \abseta jet~(\pythia8) ]{\label{fig:QG-jetHLPtb}\includegraphics[width=0.45\textwidth]{Figures/jetHLNoSkimCut_jetPtL}}
  \caption[]{
    The \pt distribution of \subref{fig:QG-jetHLPta} the higher \abseta jet %
    and \subref{fig:QG-jetHLPtb} the lower \abseta jet in multi-jet events with \pythia8 MC . %
    Data for both figures are 32.9\ifb in 2016. %
    In the low \pt range~($<100\GeV$), number of data is much smaller than the simulation due to the \pt skimming only on data in the data processing, %
    but the low \pt range is not used for the calibration.
    The normalization of the simulation is decided by cross-section.
    \label{fig:QG-jetHLPt}
    }
\end{figure}


\begin{figure}[htb]
  \centering
  \subfloat[For 2-process extraction~(\pythia8)               ]{\label{fig:QG-Fmca}\includegraphics[width=0.45\textwidth]{Figures/2sample_Fmc}} \quad
  \subfloat[For higher/lower \abseta jet extraction~(\pythia8)]{\label{fig:QG-Fmcb}\includegraphics[width=0.45\textwidth]{Figures/JetHL_Fmc}} 
  \\\subfloat[For 2-process extraction~(\sherpa)              ]{\label{fig:QG-Fmcc}\includegraphics[width=0.45\textwidth]{Figures/2sampleSherpa_Fmc}} \quad
  \subfloat[For higher/lower \abseta jet extraction~(\sherpa) ]{\label{fig:QG-Fmcd}\includegraphics[width=0.45\textwidth]{Figures/JetHLSherpa_Fmc}}
  \caption[]{
    Fractions of quark jets and gluon jets in each of \subref{fig:QG-Fmca}\subref{fig:QG-Fmcc} Z+jets and multi-jet for 2-process extraction %
    and \subref{fig:QG-Fmcb}\subref{fig:QG-Fmcd} higher \abseta jet and lower \abseta jet.~(H and L represent higher and lower \abseta jets.) %
    \pythia8 and \sherpa~MCs are used in the top and bottom figures, respectively.
    These values are used as elements in $F$ matrix in \Eqn{eq:QG-matrix}(\ref{eq:QG-matrix2}). %
    \label{fig:QG-Fmc}
    }
\end{figure}


%\begin{figure}[htb]
%  \centering
%  \subfloat[For 2-process extraction~(\pythia8)               ]{\label{fig:QG-Finversea}\includegraphics[width=0.45\textwidth]{Figures/2sample_Finverse}} \quad
%  \subfloat[For higher/lower \abseta jet extraction~(\pythia8)]{\label{fig:QG-Finverseb}\includegraphics[width=0.45\textwidth]{Figures/JetHL_Finverse}} 
%  \\\subfloat[For 2-process extraction~(\sherpa)                ]{\label{fig:QG-Finversea}\includegraphics[width=0.45\textwidth]{Figures/2sampleSherpa_Finverse}} \quad
%  \subfloat[For higher/lower \abseta jet extraction~(\sherpa) ]{\label{fig:QG-Finverseb}\includegraphics[width=0.45\textwidth]{Figures/JetHLSherpa_Finverse}}
%  \caption[]{
%    The fractions of quark jets and gluon jets in each of \subref{fig:QG-Fmca} Z+jets and multi-jet for 2-process extraction %
%    and \subref{fig:QG-Fmcb} higher \abseta jet and lower \abseta jet.~(H and L represent higher and lower \abseta jet, respectively.) %
%    These values are used as the elements in $F$ matrix in \Eqn{eq:QG-matrix}. %
%    \label{fig:QG-Fmc}
%    }
%\end{figure}


\FloatBarrier

\section{MC closure}
\label{sec:QG-closure}

%The matrix method is valid only if the shape of \wtrk in each of quark and gluon is the same between two samples %
%used in inputs of the matrix method~\Eqn{eq:QG-invmatrix}. %
The matrix method (\Eqn{eq:QG-invmatrix}) is valid only if the shape of the \wtrk distributions is the same between the quark/gluon-enriched samples, %
respectively for quark and gluon jets. %
The validation of this assumption is performed in the MC simulation by injecting the MC samples as $p_{\mrm{Q/G-rich}}(\wtrk)$ in \Eqn{eq:QG-invmatrix}. %
The difference between pure quark/gluon samples defined by the partonic flavor label in the MC %
and the extracted pure quark/gluon samples by \Eqn{eq:QG-invmatrix} is defined as an MC non-closure. %
Since there is a bit difference between Z+jets and multi-jet or between higher \abseta and lower \abseta jets in the MC~(\Figss{fig:QG-WtrkMC}), %
there is $10\%$ MC non-closure at maximum in the mean of \wtrk in both of \pythia8 and \sherpa~as shown in \Figss{fig:QG-WtrkClosure}. %
In \pythia8, the maximum MC non-closure exists in quark \wtrk in the jet \pt range between 300 and 400\GeV in the 2-process extraction. %
In this range, the gluon \wtrk in multi-jet is lower than that in Z+jets as shown in \Fig{fig:QG-WtrkMC2sample}. %
This difference causes the MC non-closure in the extraction. %
Overall, the MC non-closure in \sherpa~is larger than that in \pythia8. %
%In the matrix method, pure quark jet is obtained by the subtraction of gluon component in Z+jets by gluon component in multi-jet. %
%From this subtraction, the mean of the extracted pure quark by the matrix method has larger \wtrk than the pure quark defined by the parton label. %
%The comparison in the mean of \wtrk in each jet \pt bins between pure quark/gluon extracted by the label and that extracted by the matrix method is shown in \Figss{fig:QG-WtrkClosure}. %
This MC non-closure is taken into account as a systematic uncertainty in \Sect{sec:QG-syst}.

\begin{figure}[htb]
  \centering
  \subfloat[Z+jets v.s. Multi-jet~(\pythia8)            ]{\label{fig:QG-WtrkMC2sample}\includegraphics[width=0.45\textwidth]{Figures/2sample_MCHLcompare/jetWtrk_TightPt500_Pt0200-0300_HL_nonlog}} \quad
  \subfloat[Higher \abseta v.s. Lower \abseta~(\pythia8)]{\label{fig:QG-WtrkMCJetHL}  \includegraphics[width=0.45\textwidth]{Figures/JetHL_MCHLcompare/jetWtrk_TightPt500_Pt0500-0600_MCHLQG_nonlog}}
  \caption[]{
    Comparisons in the shape of \wtrk distributions in quark or gluon jets %
    between \subref{fig:QG-WtrkMC2sample} the leading jets of Z+jets events and multi-jet events in the jet \pt range of 200--300\GeV, and %
    between \subref{fig:QG-WtrkMCJetHL} higher \abseta and lower \abseta jets in the jet \pt range of 500--600\GeV. %
    The two samples in the comparison are equalized in the total number of events. %
    The MC is \pythia8.
    \label{fig:QG-WtrkMC}
    }
\end{figure}


\begin{figure}[htb]
  \centering
  \subfloat[For 2-process extraction~(\pythia8)               ]{\label{fig:QG-MCclosure2sample_Ntrk}\includegraphics[width=0.45\textwidth]{Figures/2sample_jetPt0B__compareProfile_Ntrk}} \quad
  \subfloat[For higher/lower \abseta jet extraction~(\pythia8)]{\label{fig:QG-MCclosureJetHL_Ntrk}  \includegraphics[width=0.45\textwidth]{Figures/JetHL_jetPtHL__compareProfile_Ntrk}}
  \\\subfloat[For 2-process extraction~(\sherpa)              ]{\label{fig:QG-MCclosure2sampleS_Ntrk}\includegraphics[width=0.45\textwidth]{Figures/2sampleSherpa_jetPt0B__compareProfile_Ntrk}} \quad
  \subfloat[For higher/lower \abseta jet extraction~(\sherpa) ]{\label{fig:QG-MCclosureJetHLS_Ntrk}  \includegraphics[width=0.45\textwidth]{Figures/JetHLSherpa_jetPtHL__compareProfile_Ntrk}}
  \caption[]{
    The difference in the mean of \ntrk between the pure quark or gluon jets defined by the partonic flavor label %
    and the extracted pure quark or gluon jets by the matrix method in each \pt bins. %
    \subref{fig:QG-MCclosure2sample_Ntrk}\subref{fig:QG-MCclosure2sampleS_Ntrk} are comparisons in 2-process extraction, 
    which is used in the lower jet \pt range below 500\GeV.
    \subref{fig:QG-MCclosureJetHL_Ntrk}\subref{fig:QG-MCclosureJetHLS_Ntrk} are comparisons in the matrix method %
    using the higher and lower \abseta jets, 
    which is used in the higher jet \pt range above 500\GeV.
    \pythia8 and \sherpa~MC simulations are used in the top and bottom figures, respectively.
    \label{fig:QG-NtrkClosure}
    }
\end{figure}



\begin{figure}[htb]
  \centering
  \subfloat[For 2-process extraction~(\pythia8)               ]{\label{fig:QG-MCclosure2sample_Wtrk}\includegraphics[width=0.45\textwidth]{Figures/2sample_jetPt0B__compareProfile_Wtrk}} \quad
  \subfloat[For higher/lower \abseta jet extraction~(\pythia8)]{\label{fig:QG-MCclosureJetHL_Wtrk}  \includegraphics[width=0.45\textwidth]{Figures/JetHL_jetPtHL__compareProfile_Wtrk}}
  \\\subfloat[For 2-process extraction~(\sherpa)              ]{\label{fig:QG-MCclosure2sampleS_Wtrk}\includegraphics[width=0.45\textwidth]{Figures/2sampleSherpa_jetPt0B__compareProfile_Wtrk}} \quad
  \subfloat[For higher/lower \abseta jet extraction~(\sherpa) ]{\label{fig:QG-MCclosureJetHLS_Wtrk}  \includegraphics[width=0.45\textwidth]{Figures/JetHLSherpa_jetPtHL__compareProfile_Wtrk}}
  \caption[]{
    The difference in the mean of \wtrk between the pure quark or gluon jets defined by the partonic flavor label %
    and the extracted pure quark or gluon jets by the matrix method in each \pt bins. %
    \subref{fig:QG-MCclosure2sample_Wtrk}\subref{fig:QG-MCclosure2sampleS_Wtrk} are comparisons in 2-process extraction, 
    which is used in the lower jet \pt range below 500\GeV.
    \subref{fig:QG-MCclosureJetHL_Wtrk}\subref{fig:QG-MCclosureJetHLS_Wtrk} are comparisons in the matrix method %
    using the higher and lower \abseta jets, 
    which is used in the higher jet \pt range above 500\GeV.
    \pythia8 and \sherpa~MC simulations are used in the top and bottom figures, respectively.
    \label{fig:QG-WtrkClosure}
    }
\end{figure}

\begin{figure}[htb]
  \centering
  \subfloat[For 2-process extraction~(\pythia8)               ]{\label{fig:QG-MCclosure2sample_C1B02}\includegraphics[width=0.45\textwidth]{Figures/2sample_jetPt0B__compareProfile_C1B02}} \quad
  \subfloat[For higher/lower \abseta jet extraction~(\pythia8)]{\label{fig:QG-MCclosureJetHL_C1B02}  \includegraphics[width=0.45\textwidth]{Figures/JetHL_jetPtHL__compareProfile_C1B02}}
  \\\subfloat[For 2-process extraction~(\sherpa)              ]{\label{fig:QG-MCclosure2sampleS_C1B02}\includegraphics[width=0.45\textwidth]{Figures/2sampleSherpa_jetPt0B__compareProfile_C1B02}} \quad
  \subfloat[For higher/lower \abseta jet extraction~(\sherpa) ]{\label{fig:QG-MCclosureJetHLS_C1B02}  \includegraphics[width=0.45\textwidth]{Figures/JetHLSherpa_jetPtHL__compareProfile_C1B02}}
  \caption[]{
    The difference in the mean of \cbeta between the pure quark or gluon jets defined by the partonic flavor label %
    and the extracted pure quark or gluon jets by the matrix method in each \pt bins. %
    \subref{fig:QG-MCclosure2sample_C1B02}\subref{fig:QG-MCclosure2sampleS_C1B02} are comparisons in 2-process extraction, 
    which is used in the lower jet \pt range below 500\GeV.
    \subref{fig:QG-MCclosureJetHL_C1B02}\subref{fig:QG-MCclosureJetHLS_C1B02} are comparisons in the matrix method %
    using the higher and lower \abseta jets, 
    which is used in the higher jet \pt range above 500\GeV.
    \pythia8 and \sherpa~MC simulations are used in the top and bottom figures, respectively.
    \label{fig:QG-C1B02Closure}
    }
\end{figure}


\FloatBarrier

\section{Scale factor}
\label{sec:QG-SF}

The extracted pure-quark/gluon distributions are shown in \Figss{fig:QG-WtrkData}. %
The "scale factor"~("SF") in each \wtrk bin and in each \pt bin for quark and gluon jets is respectively calculated from these distributions %
in order to correct the shape of \wtrk distributions. %
The SF for a quark/gluon jet with a \pt and \wtrk is given as %
\begin{equation}
  SF_{\mrm{Q/G}}\kakko{\wtrk;p_{\text{T,j}}} = \frac{ p_{\mrm{Q/G,~Ext.Data}}\kakko{\wtrk;p_{\text{T,j}}} }{ p_{\mrm{Q/G,~Ext.MC}}\kakko{\wtrk;p_{\text{T,j}}}  }, 
  \label{eq:QG-SF}
\end{equation}
where Q/G indicates quark or gluon, $p_{\text{T,j}}$ represents the $j$-th jet \pt bin, %
and $p_{\mrm{Q/G,~Ext.Data/MC}}$ is a distribution of the pure extracted quark or gluon jet from data or MC shown as in \Figss{fig:QG-WtrkData}. %
The denominator on the right-hand side is the distribution of the extracted quark or gluon jet from the MC by the same matrix method %
in order to suppress the influence of the MC non-closure. %
This SF is used as a weight for jet in the analysis when the \wtrk of the jet is used in the selection. %
Systematic uncertainties are considered as "SF up/down" with an up/down variation from the nominal SF. %

\begin{figure}[htb]
  \centering
  \subfloat[Quark: $100<\pt<150\GeV$~(\pythia8) ]{\label{fig:QG-NtrkDataQa}\includegraphics[width=0.45\textwidth]{Figures/Pythia_ShapeQuark/jetNtrk_TightPt500_Pt0100-0150uncScale100_Pythia_ShapeQuark}} \quad
  \subfloat[Gluon: $100<\pt<150\GeV$~(\pythia8) ]{\label{fig:QG-NtrkDataGa}\includegraphics[width=0.45\textwidth]{Figures/Pythia_ShapeGluon/jetNtrk_TightPt500_Pt0100-0150uncScale100_Pythia_ShapeGluon}} \\
  \subfloat[Quark: $800<\pt<1000\GeV$~(\pythia8)]{\label{fig:QG-NtrkDataQb}\includegraphics[width=0.45\textwidth]{Figures/Pythia_ShapeQuark/jetNtrk_TightPt500_Pt0800-1000uncScale100_Pythia_ShapeQuark}} \quad
  \subfloat[Gluon: $800<\pt<1000\GeV$~(\pythia8)]{\label{fig:QG-NtrkDataGb}\includegraphics[width=0.45\textwidth]{Figures/Pythia_ShapeGluon/jetNtrk_TightPt500_Pt0800-1000uncScale100_Pythia_ShapeGluon}}
  \caption[]{
    The \ntrk distributions of pure \subref{fig:QG-NtrkDataQa}\subref{fig:QG-NtrkDataQb} quark jets and %
    \subref{fig:QG-NtrkDataGa}\subref{fig:QG-NtrkDataGb} gluon jets %
    extracted by the matrix method from the data and MC. %
    The top two figures show \ntrk in the jet \pt range between 100 and 150\GeV, and %
    the bottom two figures show \ntrk in the jet \pt range between 800 and 1000\GeV. %
    Solid-line histograms show the \ntrk distributions of quark or gluon jets defined by the jet parton flavor label in the MC.
    A bottom panel in each figure shows the ratio of the extracted data to the extracted MC by the matrix method~(break line) %
    or the MC defined by the parton label~(solid line). %
    \label{fig:QG-NtrkData}
    }
\end{figure}

\begin{figure}[htb]
  \centering
  \subfloat[Quark: $100<\pt<150\GeV$~(\pythia8) ]{\label{fig:QG-WtrkDataQa}\includegraphics[width=0.45\textwidth]{Figures/Pythia_ShapeQuark/jetWtrk_TightPt500_Pt0100-0150uncScale100_Pythia_ShapeQuark}} \quad
  \subfloat[Gluon: $100<\pt<150\GeV$~(\pythia8) ]{\label{fig:QG-WtrkDataGa}\includegraphics[width=0.45\textwidth]{Figures/Pythia_ShapeGluon/jetWtrk_TightPt500_Pt0100-0150uncScale100_Pythia_ShapeGluon}} \\
  \subfloat[Quark: $800<\pt<1000\GeV$~(\pythia8)]{\label{fig:QG-WtrkDataQb}\includegraphics[width=0.45\textwidth]{Figures/Pythia_ShapeQuark/jetWtrk_TightPt500_Pt0800-1000uncScale100_Pythia_ShapeQuark}} \quad
  \subfloat[Gluon: $800<\pt<1000\GeV$~(\pythia8)]{\label{fig:QG-WtrkDataGb}\includegraphics[width=0.45\textwidth]{Figures/Pythia_ShapeGluon/jetWtrk_TightPt500_Pt0800-1000uncScale100_Pythia_ShapeGluon}}
  \caption[]{
    The \wtrk distributions of pure \subref{fig:QG-WtrkDataQa}\subref{fig:QG-WtrkDataQb} quark jets and %
    \subref{fig:QG-WtrkDataGa}\subref{fig:QG-WtrkDataGb} gluon jets %
    extracted by the matrix method from the data and MC. %
    The top two figures show \wtrk in the jet \pt range between 100 and 150\GeV, and %
    the bottom two figures show \wtrk in the jet \pt range between 800 and 1000\GeV. %
    Solid-line histograms show the \wtrk distributions of quark or gluon jets defined by the jet parton flavor label in the MC.
    A bottom panel in each figure shows the ratio of the extracted data to the extracted MC by the matrix method~(break line) %
    or the MC defined by the parton label~(solid line). %
    \label{fig:QG-WtrkData}
    }
\end{figure}

\begin{figure}[htb]
  \centering
  \subfloat[Quark: $100<\pt<150\GeV$~(\pythia8) ]{\label{fig:QG-C1B02DataQa}\includegraphics[width=0.45\textwidth]{Figures/Pythia_ShapeQuark/jetC1B02_TightPt500_Pt0100-0150uncScale100_Pythia_ShapeQuark}} \quad
  \subfloat[Gluon: $100<\pt<150\GeV$~(\pythia8) ]{\label{fig:QG-C1B02DataGa}\includegraphics[width=0.45\textwidth]{Figures/Pythia_ShapeGluon/jetC1B02_TightPt500_Pt0100-0150uncScale100_Pythia_ShapeGluon}} \\
  \subfloat[Quark: $800<\pt<1000\GeV$~(\pythia8)]{\label{fig:QG-C1B02DataQb}\includegraphics[width=0.45\textwidth]{Figures/Pythia_ShapeQuark/jetC1B02_TightPt500_Pt0800-1000uncScale100_Pythia_ShapeQuark}} \quad
  \subfloat[Gluon: $800<\pt<1000\GeV$~(\pythia8)]{\label{fig:QG-C1B02DataGb}\includegraphics[width=0.45\textwidth]{Figures/Pythia_ShapeGluon/jetC1B02_TightPt500_Pt0800-1000uncScale100_Pythia_ShapeGluon}}
  \caption[]{
    The \cbeta distributions of pure \subref{fig:QG-C1B02DataQa}\subref{fig:QG-C1B02DataQb} quark jets and %
    \subref{fig:QG-C1B02DataGa}\subref{fig:QG-C1B02DataGb} gluon jets %
    extracted by the matrix method from the data and MC. %
    The top two figures show \cbeta in the jet \pt range between 100 and 150\GeV, and %
    the bottom two figures show \cbeta in the jet \pt range between 800 and 1000\GeV. %
    Solid-line histograms show the \cbeta distributions of quark or gluon jets defined by the jet parton flavor label in the MC.
    A bottom panel in each figure shows the ratio of the extracted data to the extracted MC by the matrix method~(break line) %
    or the MC defined by the parton label~(solid line). %
    \label{fig:QG-C1B02Data}
    }
\end{figure}


\section{Systematic uncertainties}
\label{sec:QG-syst}

The systematic uncertainties are listed below:
\begin{itemize}
  \item (Parton shower modeling)
  \item The MC non-closure
  \item PDF uncertainties
  \item Tracking uncertainties
  \item Data statistical uncertainty
\end{itemize}
To avoid an additional systematic uncertainty coming from the parton shower modeling, %
this calibration is performed for \pythia8~and \sherpa~separately~as described before. %
The MC non-closure is the difference between the MC defined by the jet parton label and the extracted MC described in \Sect{sec:QG-closure}. %
The half of it is added as a systematic uncertainty to the SF up/down symmetrically. % 
PDF uncertainties are obtained by using \textsc{LHAPDF-6.1.5} package~\cite{LHAPDF}, which provides other PDF sets and the PDF internal variations for each PDF set %
as weight variations depending on the momentum fraction~$x$ and the partonic flavor of a reacted parton in the collided protons. %
By changing a nominal PDF weight to the systematic variation, %
the MC inputs with the PDF variation are obtained. %
The variation on the SF from the PDF uncertainties is calculated from the difference between the nominal SF and %
the SF computed from the MC inputs with the PDF variation weight. %
The MC inputs are $F$ matrix in \Eqns{eq:QG-matrix}{eq:QG-matrix2}, and $p_{\mrm{Q/G, Ext. MC}}$ in \Eqn{eq:QG-SF}. %
Here, \nnpdf~\cite{NNPDF30}, \ctten~\cite{CT10}, and \mmht~\cite{MMHT2014} PDF sets and their internal systematic variations are considered. %
The total up and down variations coming from all of the PDF uncertainties are determined from the envelope of the three PDF set variations. %
For the tracking uncertainties, there are five sources of systematic uncertainties: %
\begin{description}
  \item[Track reconstruction efficiency]  \mbox{} \\
    The uncertainty on the track reconstruction efficiency caused by the uncertainty of the ID material distributions, which is less than 1\% in the efficiency.
    % TRK_EFF_TIGHT_GLOBAL/IBL/PP0/PHYSMODEL
  \item[Fake track rate]  \mbox{} \\
    The uncertainty on the rate of reconstructed fake tracks passing the track ID selection. %, which is relative 27\% uncertainty. %(for LOOSE : but there is only loose fake rate systematics.)
    % TRK_FAKE_RATE_LOOSE (There is only LOOSE)
  \item[Impact parameter resolution] \mbox{} \\
    The uncertainty on the transverse~($d_0$) and longitudinal~($z_0$) impact parameter resolution. This reflects the difference in the resolution between the data and MC. %
    % TRK_RES_D0/Z0_MEAS/DEAD
  \item[Detector distortion] \mbox{} \\
    The uncertainty on the reconstructed $Q/p$~(charge over momentum), $d_0$, and $z_0$ caused by the detector distortion that is not able to be considered in the alignment of the ID. %
    % TRK_BIAS _D0/Z0_WM / _QOVERP_SAGITTA_WM
  \item[Lost track in a dense environment] \mbox{} \\
    The uncertainty on the probability of losing a track in a core of jets %
    due to the track-dense environment inside the jets. 
    % TRK_EFF_LOOSE_TIDE (There is only LOOSE)
\end{description}
These uncertainties are considered by randomly dropping a track or changing its parameter before \wtrk computation. %

The total uncertainty of the above~(quadrature sum of them) and the total systematic uncertainties on the SF using \pythia8 %
are shown in \Figss{fig:QG-SystShapePythia}, %
and also the breakdown of the systematic uncertainties for \pythia8 is shown in \Figss{fig:QG-UncPythia}. %
In \pythia8, up or down uncertainty at the peak of \wtrk is $\sim 10\%$ for both of quark and gluon jets in the \pt range between 100--150\GeV %
and for gluon jets in the \pt rage of 800--1000\GeV. It is mainly caused by the MC non-closure. %
For quark jets with \pt between 800 and 1000\GeV, the uncertainty at the peak is $\sim 5\%$ mainly contributed by the PDF uncertainties.
The uncertainties obtained by \sherpa~are shown in \Figss{fig:QG-SystShapeSherpa} and \Figss{fig:QG-UncSherpa}. %
%The $Z$+jets in the 2-process method is generated by \sherpa~2.2.1 and multi-jet events are generated by \sherpa~2.1.1. %
If \sherpa~samples are used, there is also 5--10\% uncertainties as in \pythia8. %



\begin{figure}[htb]
  \centering
  \subfloat[Quark: $100<\pt<150\GeV$~(\pythia8) ]{\label{fig:QG-SystShapePythiaQa_Ntrk}\includegraphics[width=0.45\textwidth]{Figures/Pythia_ShapeQuarkSyst/jetNtrk_TightPt500_Pt0100-0150uncScale100_Pythia_ShapeQuarkSyst}} \quad
  \subfloat[Gluon: $100<\pt<150\GeV$~(\pythia8) ]{\label{fig:QG-SystShapePythiaGa_Ntrk}\includegraphics[width=0.45\textwidth]{Figures/Pythia_ShapeGluonSyst/jetNtrk_TightPt500_Pt0100-0150uncScale100_Pythia_ShapeGluonSyst}} \\
  \subfloat[Quark: $800<\pt<1000\GeV$~(\pythia8)]{\label{fig:QG-SystShapePythiaQb_Ntrk}\includegraphics[width=0.45\textwidth]{Figures/Pythia_ShapeQuarkSyst/jetNtrk_TightPt500_Pt0800-1000uncScale100_Pythia_ShapeQuarkSyst}} \quad
  \subfloat[Gluon: $800<\pt<1000\GeV$~(\pythia8)]{\label{fig:QG-SystShapePythiaGb_Ntrk}\includegraphics[width=0.45\textwidth]{Figures/Pythia_ShapeGluonSyst/jetNtrk_TightPt500_Pt0800-1000uncScale100_Pythia_ShapeGluonSyst}}
  \caption[]{
    The \ntrk distributions of the extracted \subref{fig:QG-SystShapePythiaQa_Ntrk}\subref{fig:QG-SystShapePythiaQb_Ntrk} quark jets and %
    \subref{fig:QG-SystShapePythiaGa_Ntrk}\subref{fig:QG-SystShapePythiaGb_Ntrk} gluon jets from the data and MC %
    with the total uncertainties and total systematic uncertainties obtained by \pythia8 MCs. %
    The top two figures show the distributions in the jet \pt range between 100 and 150\GeV. %
    The bottom two show that in the jet \pt range between 800 and 1000\GeV. %
    The lower panel in each figure shows the extracted data divided by the extracted MC in the upper panels, %
    which corresponds to the \ntrk SF. %
    \label{fig:QG-SystShapePythia_Ntrk}
  }
\end{figure}


\begin{figure}[htb]
  \centering
  \subfloat[Quark : $100<\pt<150\GeV$~(\pythia8) ]{\label{fig:QG-UncPythiaQa_Ntrk}\includegraphics[width=0.5\textwidth]{Figures/Pythia_UncertaintiesQ/jetNtrk_TightPt500_Pt0100-0150uncScale100_Pythia_UncertaintiesQ}}
  \subfloat[Gluon : $100<\pt<150\GeV$~(\pythia8) ]{\label{fig:QG-UncPythiaGa_Ntrk}\includegraphics[width=0.5\textwidth]{Figures/Pythia_UncertaintiesG/jetNtrk_TightPt500_Pt0100-0150uncScale100_Pythia_UncertaintiesG}} \\
  \subfloat[Quark : $800<\pt<1000\GeV$~(\pythia8)]{\label{fig:QG-UncPythiaQb_Ntrk}\includegraphics[width=0.5\textwidth]{Figures/Pythia_UncertaintiesQ/jetNtrk_TightPt500_Pt0800-1000uncScale100_Pythia_UncertaintiesQ}}
  \subfloat[Gluon : $800<\pt<1000\GeV$~(\pythia8)]{\label{fig:QG-UncPythiaGb_Ntrk}\includegraphics[width=0.5\textwidth]{Figures/Pythia_UncertaintiesG/jetNtrk_TightPt500_Pt0800-1000uncScale100_Pythia_UncertaintiesG}}
  \caption[]{
    The fraction of each uncertainty on the jet tagging variable obtained by \pythia8 MCs as a function of \ntrk %
    for \subref{fig:QG-SystShapePythiaQa_Ntrk}\subref{fig:QG-SystShapePythiaQb_Ntrk} quark jets and %
    \subref{fig:QG-SystShapePythiaGa_Ntrk}\subref{fig:QG-SystShapePythiaGb_Ntrk} gluon jets. %
    The top two figures show the uncertainties in the jet \pt range between 100 and 150\GeV. %
    The bottom two show that in the jet \pt range between 800 and 1000\GeV. %
    The PDF up/down shows the envelope of all of the PDF uncertainties. %
    The track systematic up/down shows the quadrature sum of five sources of track systematic uncertainties. %
    \label{fig:QG-UncPythia_Ntrk}
  }
\end{figure}


\begin{figure}[htb]
  \centering
  \subfloat[Quark : $100<\pt<150\GeV$~(\sherpa) ]{\label{fig:QG-SystShapeSherpaQa}\includegraphics[width=0.45\textwidth]{Figures/Sherpa_ShapeQuarkSyst/jetNtrk_TightPt500_Pt0100-0150uncScale100_Sherpa_ShapeQuarkSyst}} \quad
  \subfloat[Gluon : $100<\pt<150\GeV$~(\sherpa) ]{\label{fig:QG-SystShapeSherpaGa}\includegraphics[width=0.45\textwidth]{Figures/Sherpa_ShapeGluonSyst/jetNtrk_TightPt500_Pt0100-0150uncScale100_Sherpa_ShapeGluonSyst}} \\
  \subfloat[Quark : $800<\pt<1000\GeV$~(\sherpa)]{\label{fig:QG-SystShapeSherpaQb}\includegraphics[width=0.45\textwidth]{Figures/Sherpa_ShapeQuarkSyst/jetNtrk_TightPt500_Pt0800-1000uncScale100_Sherpa_ShapeQuarkSyst}} \quad
  \subfloat[Gluon : $800<\pt<1000\GeV$~(\sherpa)]{\label{fig:QG-SystShapeSherpaGb}\includegraphics[width=0.45\textwidth]{Figures/Sherpa_ShapeGluonSyst/jetNtrk_TightPt500_Pt0800-1000uncScale100_Sherpa_ShapeGluonSyst}}
  \caption[]{
    The \ntrk distribution of the extracted \subref{fig:QG-SystShapePythiaQa_Ntrk}\subref{fig:QG-SystShapePythiaQb_Ntrk} quark jets and %
    \subref{fig:QG-SystShapePythiaGa_Ntrk}\subref{fig:QG-SystShapePythiaGb_Ntrk} gluon jets from data and MC %
    with the total uncertainties and the total systematic uncertainties obtained by \sherpa~MCs. %
    The top two figures show the distribution in the jet \pt range between 100 and 150\GeV. %
    The bottom two show that in the jet \pt range between 800 and 1000\GeV. %
    The lower panel in each figure shows the extracted data divided by the extracted MC in the upper panels, %
    which corresponds to the \ntrk SF. %
    \label{fig:QG-SystShapeSherpa_Ntrk}
  }
\end{figure}


\begin{figure}[htb]
  \centering
  \subfloat[Quark : $100<\pt<150\GeV$~(\sherpa) ]{\label{fig:QG-UncSherpaQa_Ntrk}\includegraphics[width=0.5\textwidth]{Figures/Method2/uncertainty/SF_plots_ntrk/quark_100_0_sherpa_SF_ntrk.pdf}}
  \subfloat[Gluon : $100<\pt<150\GeV$~(\sherpa) ]{\label{fig:QG-UncSherpaGa_Ntrk}\includegraphics[width=0.5\textwidth]{Figures/Method2/uncertainty/SF_plots_ntrk/gluon_100_0_sherpa_SF_ntrk.pdf}} \\
  \subfloat[Quark : $800<\pt<1000\GeV$~(\sherpa)]{\label{fig:QG-UncSherpaQb_Ntrk}\includegraphics[width=0.5\textwidth]{Figures/Sherpa_UncertaintiesQ/jetNtrk_TightPt500_Pt0800-1000uncScale100_Sherpa_UncertaintiesQ}}
  \subfloat[Gluon : $800<\pt<1000\GeV$~(\sherpa)]{\label{fig:QG-UncSherpaGb_Ntrk}\includegraphics[width=0.5\textwidth]{Figures/Sherpa_UncertaintiesG/jetNtrk_TightPt500_Pt0800-1000uncScale100_Sherpa_UncertaintiesG}}
  \caption[]{
    The fraction of each uncertainty on the jet tagging variable obtained by \sherpa~MCs as a function of \ntrk %
    for \subref{fig:QG-SystShapePythiaQa_Ntrk}\subref{fig:QG-SystShapePythiaQb_Ntrk} quark jets and %
    \subref{fig:QG-SystShapePythiaGa_Ntrk}\subref{fig:QG-SystShapePythiaGb_Ntrk} gluon jets. %
    The top two figures show the uncertainties in the jet \pt range between 100 and 150\GeV. %
    The bottom two show that in the jet \pt range between 800 and 1000\GeV. %
    The PDF up/down shows the envelope of all of the PDF uncertainties. %
    The track systematic up/down shows the quadrature sum of five sources of track systematic uncertainties. %
    \label{fig:QG-UncSherpa_Ntrk}
  }
\end{figure}




\begin{figure}[htb]
  \centering
  \subfloat[Quark: $100<\pt<150\GeV$~(\pythia8) ]{\label{fig:QG-SystShapePythiaQa_Wtrk}\includegraphics[width=0.45\textwidth]{Figures/Pythia_ShapeQuarkSyst/jetWtrk_TightPt500_Pt0100-0150uncScale100_Pythia_ShapeQuarkSyst}} \quad
  \subfloat[Gluon: $100<\pt<150\GeV$~(\pythia8) ]{\label{fig:QG-SystShapePythiaGa_Wtrk}\includegraphics[width=0.45\textwidth]{Figures/Pythia_ShapeGluonSyst/jetWtrk_TightPt500_Pt0100-0150uncScale100_Pythia_ShapeGluonSyst}} \\
  \subfloat[Quark: $800<\pt<1000\GeV$~(\pythia8)]{\label{fig:QG-SystShapePythiaQb_Wtrk}\includegraphics[width=0.45\textwidth]{Figures/Pythia_ShapeQuarkSyst/jetWtrk_TightPt500_Pt0800-1000uncScale100_Pythia_ShapeQuarkSyst}} \quad
  \subfloat[Gluon: $800<\pt<1000\GeV$~(\pythia8)]{\label{fig:QG-SystShapePythiaGb_Wtrk}\includegraphics[width=0.45\textwidth]{Figures/Pythia_ShapeGluonSyst/jetWtrk_TightPt500_Pt0800-1000uncScale100_Pythia_ShapeGluonSyst}}
  \caption[]{
    The \wtrk distributions of the extracted \subref{fig:QG-SystShapePythiaQa_Wtrk}\subref{fig:QG-SystShapePythiaQb_Wtrk} quark jets and %
    \subref{fig:QG-SystShapePythiaGa_Wtrk}\subref{fig:QG-SystShapePythiaGb_Wtrk} gluon jets from the data and MC %
    with the total uncertainties and total systematic uncertainties obtained by \pythia8 MCs. %
    The top two figures show the distributions in the jet \pt range between 100 and 150\GeV. %
    The bottom two show that in the jet \pt range between 800 and 1000\GeV. %
    The lower panel in each figure shows the extracted data divided by the extracted MC in the upper panels, %
    which corresponds to the \wtrk SF. %
    \label{fig:QG-SystShapePythia_Wtrk}
  }
\end{figure}


\begin{figure}[htb]
  \centering
  \subfloat[Quark : $100<\pt<150\GeV$~(\pythia8) ]{\label{fig:QG-UncPythiaQa_Wtrk}\includegraphics[width=0.5\textwidth]{Figures/Pythia_UncertaintiesQ/jetWtrk_TightPt500_Pt0100-0150uncScale100_Pythia_UncertaintiesQ}}
  \subfloat[Gluon : $100<\pt<150\GeV$~(\pythia8) ]{\label{fig:QG-UncPythiaGa_Wtrk}\includegraphics[width=0.5\textwidth]{Figures/Pythia_UncertaintiesG/jetWtrk_TightPt500_Pt0100-0150uncScale100_Pythia_UncertaintiesG}} \\
  \subfloat[Quark : $800<\pt<1000\GeV$~(\pythia8)]{\label{fig:QG-UncPythiaQb_Wtrk}\includegraphics[width=0.5\textwidth]{Figures/Pythia_UncertaintiesQ/jetWtrk_TightPt500_Pt0800-1000uncScale100_Pythia_UncertaintiesQ}}
  \subfloat[Gluon : $800<\pt<1000\GeV$~(\pythia8)]{\label{fig:QG-UncPythiaGb_Wtrk}\includegraphics[width=0.5\textwidth]{Figures/Pythia_UncertaintiesG/jetWtrk_TightPt500_Pt0800-1000uncScale100_Pythia_UncertaintiesG}}
  \caption[]{
    The fraction of each uncertainty on the jet tagging variable obtained by \pythia8 MCs as a function of \wtrk %
    for \subref{fig:QG-SystShapePythiaQa_Wtrk}\subref{fig:QG-SystShapePythiaQb_Wtrk} quark jets and %
    \subref{fig:QG-SystShapePythiaGa_Wtrk}\subref{fig:QG-SystShapePythiaGb_Wtrk} gluon jets. %
    The top two figures show the uncertainties in the jet \pt range between 100 and 150\GeV. %
    The bottom two show that in the jet \pt range between 800 and 1000\GeV. %
    The PDF up/down shows the envelope of all of the PDF uncertainties. %
    The track systematic up/down shows the quadrature sum of five sources of track systematic uncertainties. %
    \label{fig:QG-UncPythia_Wtrk}
  }
\end{figure}


\begin{figure}[htb]
  \centering
  \subfloat[Quark : $100<\pt<150\GeV$~(\sherpa) ]{\label{fig:QG-SystShapeSherpaQa}\includegraphics[width=0.45\textwidth]{Figures/Sherpa_ShapeQuarkSyst/jetWtrk_TightPt500_Pt0100-0150uncScale100_Sherpa_ShapeQuarkSyst}} \quad
  \subfloat[Gluon : $100<\pt<150\GeV$~(\sherpa) ]{\label{fig:QG-SystShapeSherpaGa}\includegraphics[width=0.45\textwidth]{Figures/Sherpa_ShapeGluonSyst/jetWtrk_TightPt500_Pt0100-0150uncScale100_Sherpa_ShapeGluonSyst}} \\
  \subfloat[Quark : $800<\pt<1000\GeV$~(\sherpa)]{\label{fig:QG-SystShapeSherpaQb}\includegraphics[width=0.45\textwidth]{Figures/Sherpa_ShapeQuarkSyst/jetWtrk_TightPt500_Pt0800-1000uncScale100_Sherpa_ShapeQuarkSyst}} \quad
  \subfloat[Gluon : $800<\pt<1000\GeV$~(\sherpa)]{\label{fig:QG-SystShapeSherpaGb}\includegraphics[width=0.45\textwidth]{Figures/Sherpa_ShapeGluonSyst/jetWtrk_TightPt500_Pt0800-1000uncScale100_Sherpa_ShapeGluonSyst}}
  \caption[]{
    The \wtrk distribution of the extracted \subref{fig:QG-SystShapePythiaQa_Wtrk}\subref{fig:QG-SystShapePythiaQb_Wtrk} quark jets and %
    \subref{fig:QG-SystShapePythiaGa_Wtrk}\subref{fig:QG-SystShapePythiaGb_Wtrk} gluon jets from data and MC %
    with the total uncertainties and the total systematic uncertainties obtained by \sherpa~MCs. %
    The top two figures show the distribution in the jet \pt range between 100 and 150\GeV. %
    The bottom two show that in the jet \pt range between 800 and 1000\GeV. %
    The lower panel in each figure shows the extracted data divided by the extracted MC in the upper panels, %
    which corresponds to the \wtrk SF. %
    \label{fig:QG-SystShapeSherpa_Wtrk}
  }
\end{figure}


\begin{figure}[htb]
  \centering
  \subfloat[Quark : $100<\pt<150\GeV$~(\sherpa) ]{\label{fig:QG-UncSherpaQa_Wtrk}\includegraphics[width=0.5\textwidth]{Figures/Sherpa_UncertaintiesQ/jetWtrk_TightPt500_Pt0100-0150uncScale100_Sherpa_UncertaintiesQ}}
  \subfloat[Gluon : $100<\pt<150\GeV$~(\sherpa) ]{\label{fig:QG-UncSherpaGa_Wtrk}\includegraphics[width=0.5\textwidth]{Figures/Sherpa_UncertaintiesG/jetWtrk_TightPt500_Pt0100-0150uncScale100_Sherpa_UncertaintiesG}} \\
  \subfloat[Quark : $800<\pt<1000\GeV$~(\sherpa)]{\label{fig:QG-UncSherpaQb_Wtrk}\includegraphics[width=0.5\textwidth]{Figures/Sherpa_UncertaintiesQ/jetWtrk_TightPt500_Pt0800-1000uncScale100_Sherpa_UncertaintiesQ}}
  \subfloat[Gluon : $800<\pt<1000\GeV$~(\sherpa)]{\label{fig:QG-UncSherpaGb_Wtrk}\includegraphics[width=0.5\textwidth]{Figures/Sherpa_UncertaintiesG/jetWtrk_TightPt500_Pt0800-1000uncScale100_Sherpa_UncertaintiesG}}
  \caption[]{
    The fraction of each uncertainty on the jet tagging variable obtained by \sherpa~MCs as a function of \wtrk %
    for \subref{fig:QG-SystShapePythiaQa_Wtrk}\subref{fig:QG-SystShapePythiaQb_Wtrk} quark jets and %
    \subref{fig:QG-SystShapePythiaGa_Wtrk}\subref{fig:QG-SystShapePythiaGb_Wtrk} gluon jets. %
    The top two figures show the uncertainties in the jet \pt range between 100 and 150\GeV. %
    The bottom two show that in the jet \pt range between 800 and 1000\GeV. %
    The PDF up/down shows the envelope of all of the PDF uncertainties. %
    The track systematic up/down shows the quadrature sum of five sources of track systematic uncertainties. %
    \label{fig:QG-UncSherpa_Wtrk}
  }
\end{figure}


\begin{figure}[htb]
  \centering
  \subfloat[Quark: $100<\pt<150\GeV$~(\pythia8) ]{\label{fig:QG-SystShapePythiaQa_C1B02}\includegraphics[width=0.45\textwidth]{Figures/Pythia_ShapeQuarkSyst/jetC1B02_TightPt500_Pt0100-0150uncScale100_Pythia_ShapeQuarkSyst}} \quad
  \subfloat[Gluon: $100<\pt<150\GeV$~(\pythia8) ]{\label{fig:QG-SystShapePythiaGa_C1B02}\includegraphics[width=0.45\textwidth]{Figures/Pythia_ShapeGluonSyst/jetC1B02_TightPt500_Pt0100-0150uncScale100_Pythia_ShapeGluonSyst}} \\
  \subfloat[Quark: $800<\pt<1000\GeV$~(\pythia8)]{\label{fig:QG-SystShapePythiaQb_C1B02}\includegraphics[width=0.45\textwidth]{Figures/Pythia_ShapeQuarkSyst/jetC1B02_TightPt500_Pt0800-1000uncScale100_Pythia_ShapeQuarkSyst}} \quad
  \subfloat[Gluon: $800<\pt<1000\GeV$~(\pythia8)]{\label{fig:QG-SystShapePythiaGb_C1B02}\includegraphics[width=0.45\textwidth]{Figures/Pythia_ShapeGluonSyst/jetC1B02_TightPt500_Pt0800-1000uncScale100_Pythia_ShapeGluonSyst}}
  \caption[]{
    The \cbeta distributions of the extracted \subref{fig:QG-SystShapePythiaQa_C1B02}\subref{fig:QG-SystShapePythiaQb_C1B02} quark jets and %
    \subref{fig:QG-SystShapePythiaGa_C1B02}\subref{fig:QG-SystShapePythiaGb_C1B02} gluon jets from the data and MC %
    with the total uncertainties and total systematic uncertainties obtained by \pythia8 MCs. %
    The top two figures show the distributions in the jet \pt range between 100 and 150\GeV. %
    The bottom two show that in the jet \pt range between 800 and 1000\GeV. %
    The lower panel in each figure shows the extracted data divided by the extracted MC in the upper panels, %
    which corresponds to the \cbeta SF. %
    \label{fig:QG-SystShapePythia_C1B02}
  }
\end{figure}


\begin{figure}[htb]
  \centering
  \subfloat[Quark : $100<\pt<150\GeV$~(\pythia8) ]{\label{fig:QG-UncPythiaQa_C1B02}\includegraphics[width=0.5\textwidth]{Figures/Pythia_UncertaintiesQ/jetC1B02_TightPt500_Pt0100-0150uncScale100_Pythia_UncertaintiesQ}}
  \subfloat[Gluon : $100<\pt<150\GeV$~(\pythia8) ]{\label{fig:QG-UncPythiaGa_C1B02}\includegraphics[width=0.5\textwidth]{Figures/Pythia_UncertaintiesG/jetC1B02_TightPt500_Pt0100-0150uncScale100_Pythia_UncertaintiesG}} \\
  \subfloat[Quark : $800<\pt<1000\GeV$~(\pythia8)]{\label{fig:QG-UncPythiaQb_C1B02}\includegraphics[width=0.5\textwidth]{Figures/Pythia_UncertaintiesQ/jetC1B02_TightPt500_Pt0800-1000uncScale100_Pythia_UncertaintiesQ}}
  \subfloat[Gluon : $800<\pt<1000\GeV$~(\pythia8)]{\label{fig:QG-UncPythiaGb_C1B02}\includegraphics[width=0.5\textwidth]{Figures/Pythia_UncertaintiesG/jetC1B02_TightPt500_Pt0800-1000uncScale100_Pythia_UncertaintiesG}}
  \caption[]{
    The fraction of each uncertainty on the jet tagging variable obtained by \pythia8 MCs as a function of \cbeta %
    for \subref{fig:QG-SystShapePythiaQa_C1B02}\subref{fig:QG-SystShapePythiaQb_C1B02} quark jets and %
    \subref{fig:QG-SystShapePythiaGa_C1B02}\subref{fig:QG-SystShapePythiaGb_C1B02} gluon jets. %
    The top two figures show the uncertainties in the jet \pt range between 100 and 150\GeV. %
    The bottom two show that in the jet \pt range between 800 and 1000\GeV. %
    The PDF up/down shows the envelope of all of the PDF uncertainties. %
    The track systematic up/down shows the quadrature sum of five sources of track systematic uncertainties. %
    \label{fig:QG-UncPythia_C1B02}
  }
\end{figure}


\begin{figure}[htb]
  \centering
  \subfloat[Quark : $100<\pt<150\GeV$~(\sherpa) ]{\label{fig:QG-SystShapeSherpaQa}\includegraphics[width=0.45\textwidth]{Figures/Sherpa_ShapeQuarkSyst/jetC1B02_TightPt500_Pt0100-0150uncScale100_Sherpa_ShapeQuarkSyst}} \quad
  \subfloat[Gluon : $100<\pt<150\GeV$~(\sherpa) ]{\label{fig:QG-SystShapeSherpaGa}\includegraphics[width=0.45\textwidth]{Figures/Sherpa_ShapeGluonSyst/jetC1B02_TightPt500_Pt0100-0150uncScale100_Sherpa_ShapeGluonSyst}} \\
  \subfloat[Quark : $800<\pt<1000\GeV$~(\sherpa)]{\label{fig:QG-SystShapeSherpaQb}\includegraphics[width=0.45\textwidth]{Figures/Sherpa_ShapeQuarkSyst/jetC1B02_TightPt500_Pt0800-1000uncScale100_Sherpa_ShapeQuarkSyst}} \quad
  \subfloat[Gluon : $800<\pt<1000\GeV$~(\sherpa)]{\label{fig:QG-SystShapeSherpaGb}\includegraphics[width=0.45\textwidth]{Figures/Sherpa_ShapeGluonSyst/jetC1B02_TightPt500_Pt0800-1000uncScale100_Sherpa_ShapeGluonSyst}}
  \caption[]{
    The \cbeta distribution of the extracted \subref{fig:QG-SystShapePythiaQa_C1B02}\subref{fig:QG-SystShapePythiaQb_C1B02} quark jets and %
    \subref{fig:QG-SystShapePythiaGa_C1B02}\subref{fig:QG-SystShapePythiaGb_C1B02} gluon jets from data and MC %
    with the total uncertainties and the total systematic uncertainties obtained by \sherpa~MCs. %
    The top two figures show the distribution in the jet \pt range between 100 and 150\GeV. %
    The bottom two show that in the jet \pt range between 800 and 1000\GeV. %
    The lower panel in each figure shows the extracted data divided by the extracted MC in the upper panels, %
    which corresponds to the \cbeta SF. %
    \label{fig:QG-SystShapeSherpa_C1B02}
  }
\end{figure}


\begin{figure}[htb]
  \centering
  \subfloat[Quark : $100<\pt<150\GeV$~(\sherpa) ]{\label{fig:QG-UncSherpaQa_C1B02}\includegraphics[width=0.5\textwidth]{Figures/Sherpa_UncertaintiesQ/jetC1B02_TightPt500_Pt0100-0150uncScale100_Sherpa_UncertaintiesQ}}
  \subfloat[Gluon : $100<\pt<150\GeV$~(\sherpa) ]{\label{fig:QG-UncSherpaGa_C1B02}\includegraphics[width=0.5\textwidth]{Figures/Sherpa_UncertaintiesG/jetC1B02_TightPt500_Pt0100-0150uncScale100_Sherpa_UncertaintiesG}} \\
  \subfloat[Quark : $800<\pt<1000\GeV$~(\sherpa)]{\label{fig:QG-UncSherpaQb_C1B02}\includegraphics[width=0.5\textwidth]{Figures/Sherpa_UncertaintiesQ/jetC1B02_TightPt500_Pt0800-1000uncScale100_Sherpa_UncertaintiesQ}}
  \subfloat[Gluon : $800<\pt<1000\GeV$~(\sherpa)]{\label{fig:QG-UncSherpaGb_C1B02}\includegraphics[width=0.5\textwidth]{Figures/Sherpa_UncertaintiesG/jetC1B02_TightPt500_Pt0800-1000uncScale100_Sherpa_UncertaintiesG}}
  \caption[]{
    The fraction of each uncertainty on the jet tagging variable obtained by \sherpa~MCs as a function of \cbeta %
    for \subref{fig:QG-SystShapePythiaQa_C1B02}\subref{fig:QG-SystShapePythiaQb_C1B02} quark jets and %
    \subref{fig:QG-SystShapePythiaGa_C1B02}\subref{fig:QG-SystShapePythiaGb_C1B02} gluon jets. %
    The top two figures show the uncertainties in the jet \pt range between 100 and 150\GeV. %
    The bottom two show that in the jet \pt range between 800 and 1000\GeV. %
    The PDF up/down shows the envelope of all of the PDF uncertainties. %
    The track systematic up/down shows the quadrature sum of five sources of track systematic uncertainties. %
    \label{fig:QG-UncSherpa_C1B02}
  }
\end{figure}




\begin{comment}

For \pythia8, the obtained calibrations are valid only for quark and gluon jets. %
For \sherpa~in the lower \pt ranges, the calibrations are valid only for quark jets in \sherpa2.2.1 and gluon jets in \sherpa2.1.1 %
because the quark and gluon enriched samples are \sherpa2.2.1 Z+jets MC and \sherpa2.1.1 multi-jet MC, respectively. %
In the higher \pt ranges, both of the quarks and gluons are valid only for \sherpa2.1.1. %
Therefore, the calibrations cannot be used for $b$-quark jets and jets labeled "other" and also for the other parton shower modelings. %
Hence, for the other parton flavors~($b$-quark or label of "other") or the other shower modelings, %
only the uncertainties are defined and the nominal SF is set to 1. %
The uncertainty for the $b$-quarks is defined by the envelope of the uncertainties for the quarks and gluons.
The uncertainty for a different-version shower modeling~(\pythia6 for \pythia8, \sherpa2.2.1 for \sherpa2.1.1, and \sherpa2.1.1 for \sherpa2.2.1) %
is defined by the envelope of 1 and SF up/down of the obtained calibration for the corresponding shower modeling~(\pythia8 or \sherpa). %
The computation algorithms for the calibrations and the uncertainties on \wtrk SF %
for each parton flavor and each parton shower modelings %
are summarized in \Tab{tab:QG-SF}. %


\begin{table}[htb]
  \centering
  \caption{
    The summary of computation algorithms of \wtrk scale factor~(SF) and its up/down variations. %
    P or S means the value obtained in the calibration for \pythia8 or \sherpa~ is used. %
    In case that nominal SF is 1 and SF up/down is P or S, the SF up and down are obtained %
    from the envelope of 1 and SF up/down of the obtained calibration for \pythia8 or \sherpa. %
    The Q or G means the SF up and down is defined by the envelope of the SF ups and SF downs calculated for quark jets and gluon jets. %
    The P or S means the SF up and down is defined by the envelope of the SF ups and SF downs calculated with \pythia8 and \sherpa. %
    The jets labeled "other" have 100\% uncertainties on the SF.
  }
  \begin{tabular}{ll|c|c|c|c}
    \hline
    \multicolumn{2}{l|}{Parton flavor}          & Quark  & Gluon & $b$-quark         & Other   \\
    \hline                                                                               
    \multirow{2}{*}{\pythia8}  & Nominal SF & P  & P &  1.               & 1.       \\ \cdashline{2-6}
                               & SF up/down & P  & P &  Q or G & 2/0      \\ \hline
    \multirow{2}{*}{\pythia6}  & Nominal SF & 1.     & 1.    &  1.               & 1.       \\ \cdashline{2-6}
                               & SF up/down & P  & P &  Q or G & 2/0      \\ \hline
    \sherpa~2.2.1              & Nominal SF & S  & 1.    &  1.               & 1.       \\ \cdashline{2-6}
    (In the lower \pt ranges)  & SF up/down & S  & S &  Q or G & 2/0      \\ \hline
    \sherpa~2.1.1              & Nominal SF & 1.     & S &  1.               & 1.       \\ \cdashline{2-6}
    (In the lower \pt ranges)  & SF up/down & S  & S &  Q or G & 2/0      \\ \hline
    \sherpa~2.2.1              & Nominal SF & 1.     & 1.    &  1.               & 1.       \\ \cdashline{2-6}
    (In the higher \pt ranges) & SF up/down & S  & S &  Q or G & 2/0      \\ \hline
    \sherpa~2.1.1              & Nominal SF & S  & S &  1.               & 1.       \\ \cdashline{2-6}
    (In the higher \pt ranges) & SF up/down & S  & S &  Q or G & 2/0      \\ \hline
    \multirow{2}{*}{Others}    & Nominal SF & 1.     & 1.    &  1.               & 1.       \\ \cdashline{2-6}
                               & SF up/down & P or S & P or S & Q or G $\times$ P or S & 2/0 \\ 
    \hline
  \end{tabular}
  \label{tab:QG-SF}
\end{table}

\FloatBarrier

\end{comment}
